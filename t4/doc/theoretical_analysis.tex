\section{Theoretical Analysis}
\label{sec:theoretical analysis}
%%%%%%%%%%%%%%%%%%%%%%%%%%%%%%%%%%%%%%%%%%%%%

\subsection{Operating Point}

\indent

\textbf{First transistor (npn)} 

\indent

The current through the base of the first transistor is given by $$I_{B1}=(V_{eq}-V_{BEon})/(R_B+(1+\beta_{FN})\cdot R_{E1}$$

The current through the collector of the first transistor is given by $$I_{C1}=\beta _{FN}\cdot I_{B1}$$

The current through the emitter of the first transistor is given by $$I_{E1}=(1+\beta _{FN})\cdot I_{B1}$$

The voltage drop from the emitter to the collector of the first transistor is given by $$ V_{EC1}=V_{O1}-R_{E1}\cdot I_{E1} $$

The voltage drop from the emitter to the base of the first transistor is given by $$ V_{EB1} = V_{EBon} $$

By applying the previous formulas, one can obtain
\vspace{0.5cm}

\csvautotabular{../mat/resultados1_i_op.txt}

\vspace{0.5cm}

\csvautotabular{../mat/resultados1_v_op.txt}

\vspace{0.5cm}

NGSpice results

\vspace{0.5cm}

\csvautotabular{../sim/V1.txt}

\vspace{0.5cm}

\textbf{Second transistor (pnp)}

\indent



The current through the base of the first transistor is given by $$I_{B2} = I_{E2}-I_{C2}$$

The current through the collector of the first transistor is given by $$\beta_{FP}/(\beta_{FP}+1)\cdot I_{E2}$$

The current through the emitter of the first transistor is given by $$I_{E2} = (V_{CC}-V_{EBon}-V_{I2})/R_{E2}$$

The voltage drop from the emitter to the collector is given by $$ V_{EC2} = V_{O} $$

The voltage drop from the emitter to the base is given by $$ V_{EB2} = V_{EBon} $$

\csvautotabular{../mat/resultados2_i_op.txt}

\vspace{0.5cm}

\csvautotabular{../mat/resultados2_v_op.txt}

\vspace{0.5cm}

NGSpice results

\vspace{0.5cm}

\csvautotabular{../sim/V2.txt}

\vspace{0.5cm}

\textbf{Comparation of OP} 

\indent

The values of $V_{EB}$ are similar because it was an approximation of experimental results.

The values of $V_{EC}$ are not as colse as the previous one, but the order of magnitude is the same in both cases.

Furthermore, in both cases, it is possible to conclude that the transistors are working in forward active region (F.A.R.), since $V_{CE} > V_{BE}$ and $V_{EC} > V_{EB}$.





\indent



\subsection{Gain, Input and Output Impedances}

The input and output impedances of the first transistor is given by, respectively,

$$Z_{i1} = \frac {1}{\frac {1/R_b+1}{((r_{o1}+R_{c1}+R_{e1})\cdot (r_{\pi 1}+R_{e1})+g_{m1}\cdot R_{e1}\cdot r_{o1}\cdot r_{\pi 1} - R_{e1}^2)/(r_{o1}+R_{c1}+R_{e1})}}$$

$$Z_{o1} = \frac {1}{1/r_{o1}+1/R_{c1}}$$

The gain of the first transistor is given by

$$gain_1=R_{sb}/R_{s}\cdot R_{c1}\cdot \frac {R_{e1}-g_{m1}\cdot r_{\pi 1}\cdot r_{o1}}{(r_{o1}+R_{c1}+R_{e1})\cdot (R_{sb}+r_{\pi 1}+R_{e1})+g_{m1}\cdot R_{e1}\cdot r_{o1}\cdot r_{\pi 1} - R_{e1}^2} $$

The input and output impedances of the second transistor can be given by, respectively,

$$Z_{i2} = \frac {g_{m2}+g_{\pi 2}+g_{o2}+g_{e2}}{g_{\pi 2}/(g_{\pi 2}+g_{o2}+g_{e2}}$$

$$Z_{o2} = 1/(g_{m2}+g_{\pi 2}+g_{o2}+g_{e2})$$

The gain of the second transistor is given by

$$gain_2 =\frac { g_{m2}}{g_{m2}+g_{\pi 2}+g_{o2}+g_{e2}}$$

This yields,

\csvautotabular{../mat/imp.txt}

\vspace{1cm}

Because the input impedance is 161 times bigger then the ouput impedance, they can be connected without significant signal loss.



\subsection{Frequency Response}

\textbf{Gain Stage} 

 The results differ a little bit in the first stage of the graphic and are similar after that, until it starts to decrease.
 
 The reason for that is that it is assumed that for lower frequencies the capacitor in series with $V_in$ acts as a short circuit when in reality it is not.
 
 Nevertheless, the reason where the gain is increasing is approximately the same which is good.
 
 The decrease at the end is not predicted is the theoretical model because it happens due to capacitors inside the transistors which are not being considered.

\begin{figure}[h]
    \centering
\subfloat[Simulation Results]{\includegraphics[width=0.40\textwidth]{../sim/vo1f.pdf}
\label{Fig_2_estatica}}
  \hfill
\subfloat[Theoretical Results]{\includegraphics[width=0.55\textwidth]{../mat/gain1.png}
\label{Fig_2_dinamica}}
\caption{Gain stage's output voltage gain.}
\end{figure}

  
 
 \vspace{1cm}
 
 \textbf{Entire Circuit} 
 
 \indent

To analyse the gain of the entire circuit, we used the set of equations discussed in one of the posts in forum (https://groups.google.com/u/1/a/tecnico.ulisboa.pt/g/teoria-dos-circuitos-e-fundamentos-de-electrnica-tcfe/c/2oJaeUUFsTk), that can be obtained by using mesh analysis in the bellow circuit, in order to get a more accurate result

%%%%%%%%%%%%%%%%%%%%%%%%%%%%%%%%


As it can be seen bellow, this was achieved and the theoretical model and the NGSpice one are really similar.

Once again, for higher frequencies the results differ because the capacitors inside the transistor are not being considered.
 

 
 
 
 
 
 
 
 
 
 
 \begin{figure}[h]
    \centering
\subfloat[Simulation Results]{\includegraphics[width=0.40\textwidth]{../sim/vo2f.pdf}
\label{Fig_2_estatica}}
  \hfill
\subfloat[Theoretical Results]{\includegraphics[width=0.55\textwidth]{../mat/gain2.png}
\label{Fig_2_dinamica}}
\caption{Output stage's output voltage gain.}
\end{figure}

  
 
 
 

