\section{Conclusion}
\label{sec:conclusion}

\indent

Despite some differences, the results can be said to be similar.

Furthermore, we believe that the differences are not that significant and they can be explained by how NGSpice solves the circuit compared to how it was done in the theoretical analysis.

To solve the circuit, NGSpice used far more advanced simulation methods for the OP-AMP, with many more parameters. 

This way, the objective should have never been to have equal results, but rather, have results that are "close enough", which we believe it was achieved.

Furthermore, one detrimental goal was to analyse and choose values for the circuit's elements that assure it's best and most optimal performance, which by taking into consideration the obtained data, we believe it was achieved.

We would also like to add that this lab enabled us to have the experience of going to the lab and build the real circuit that it was used. It showcased the advances of using simulators like NGSpice over the physical circuit, because they solve the problem of havinf broken pieces and allow much more flexibily in the analysis. Nevertheless, we still think that is crucial to have the experience of going to the lab if building the circuits "by hand" ,in order to learn how to work with real equipment and how to deal with real world problems. 

%To sum up, we believe that the goals of this report were achieved.
